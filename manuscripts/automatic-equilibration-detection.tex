% PRL look and style (easy on the eyes)
\documentclass[aps,pre,twocolumn,nofootinbib,superscriptaddress,linenumbers]{revtex4-1}
% Two-column style (for submission/review/editing)
%\documentclass[aps,prl,preprint,nofootinbib,superscriptaddress,linenumbers]{revtex4-1}

%\usepackage{palatino}

% Change to a sans serif font.
\usepackage{helvet}
\renewcommand*\familydefault{\sfdefault} %% Only if the base font of the document is to be sans serif
\usepackage[T1]{fontenc}

\usepackage{amsmath}
\usepackage{amssymb}
\usepackage{graphicx}
%\usepackage[mathbf,mathcal]{euler}
%\usepackage{citesort}
\usepackage{dcolumn}
\usepackage{boxedminipage}
\usepackage{verbatim}
\usepackage[colorlinks=true,citecolor=blue,linkcolor=blue]{hyperref}


% The figures are in a figures/ subdirectory.
\graphicspath{{figures/}}


% italicized boldface for math (e.g. vectors)
\newcommand{\bfv}[1]{{\mbox{\boldmath{$#1$}}}}
% non-italicized boldface for math (e.g. matrices)
\newcommand{\bfm}[1]{{\bf #1}}          

%\newcommand{\bfm}[1]{{\mbox{\boldmath{$#1$}}}}
%\newcommand{\bfm}[1]{{\bf #1}}
\newcommand{\expect}[1]{\left \langle #1 \right \rangle}                % <.> for denoting expectations over realizations of an experiment or thermal averages

% vectors
\newcommand{\x}{\bfv{x}}
\newcommand{\y}{\bfv{y}}
\newcommand{\f}{\bfv{f}}

\newcommand{\bfc}{\bfm{c}}
\newcommand{\hatf}{\hat{f}}

\newcommand{\bTheta}{\bfm{\Theta}}
\newcommand{\btheta}{\bfm{\theta}}
\newcommand{\bhatf}{\bfm{\hat{f}}}
\newcommand{\Cov}[1] {\mathrm{cov}\left( #1 \right)}
\newcommand{\Ept}[1] {{\mathrm E}\left[ #1 \right]}
\newcommand{\Eptk}[2] {{\mathrm E}\left[ #2 \,|\, #1\right]}
\newcommand{\T}{\mathrm{T}}                                % T used in matrix transpose
\newcommand{\conc}[1] {\left[ \mathrm{#1} \right]}

\newcommand{\pyitc}{\url{http://www.simtk.org/home/bayesian-itc}} % URL of pyITC project homepage

%%%%%%%%%%%%%%%%%%%%%%%%%%%%%%%%%%%%%%%%%%%%%%%%%%%%%%%%%%%%%%%%%%%%%%%%%%%%%%%%
% DOCUMENT
%%%%%%%%%%%%%%%%%%%%%%%%%%%%%%%%%%%%%%%%%%%%%%%%%%%%%%%%%%%%%%%%%%%%%%%%%%%%%%%%

\begin{document}

%%%%%%%%%%%%%%%%%%%%%%%%%%%%%%%%%%%%%%%%%%%%%%%%%%%%%%%%%%%%%%%%%%%%%%%%%%%%%%%%
% TITLE
%%%%%%%%%%%%%%%%%%%%%%%%%%%%%%%%%%%%%%%%%%%%%%%%%%%%%%%%%%%%%%%%%%%%%%%%%%%%%%%%

\title{A simple method for automated equilibration detection in molecular simulations}

\author{John D. Chodera}
 \thanks{Corresponding author}
 \email{john.chodera@choderalab.org}
 \affiliation{Computational Biology Program, Sloan Kettering Institute, Memorial Sloan Kettering Cancer Center, New York, NY 10065}

\date{\today}

%%%%%%%%%%%%%%%%%%%%%%%%%%%%%%%%%%%%%%%%%%%%%%%%%%%%%%%%%%%%%%%%%%%%%%%%%%%%%%%%
% ABSTRACT
%%%%%%%%%%%%%%%%%%%%%%%%%%%%%%%%%%%%%%%%%%%%%%%%%%%%%%%%%%%%%%%%%%%%%%%%%%%%%%%%

\begin{abstract}

Molecular simulations (molecular dynamics, Monte Carlo) are often initiated from configurations that are highly dissimilar to equilibrium samples, a practice which causes the appearance of a distinct initial transient in various mechanical observables computed over the timecourse of the simulation.
Traditional practice in simulation data analysis recommends this initial transient portion be discarded as \emph{equilibration}, but no simple, general automated procedure exists.
Here, we consider a simple, automated, easy-to-implement procedure in which the final region of the simulation that maximizes the number of effectively uncorrelated samples is used.
We present a simple reference Python implementation of this procedure and illustrate its application to both synthetic and real simulation data.\\

% KEYWORDS
\emph{Keywords: molecular dynamics (MD); Monte Carlo (MC); Markov chain Monte Carlo (MCMC); equilibration; timeseries analysis; statistical inefficiency; integrated autocorrelation time}

\end{abstract}

\maketitle

%%%%%%%%%%%%%%%%%%%%%%%%%%%%%%%%%%%%%%%%%%%%%%%%%%%%%%%%%%%%%%%%%%%%%%%%%%%%%%%%
% INTRODUCTION
%%%%%%%%%%%%%%%%%%%%%%%%%%%%%%%%%%%%%%%%%%%%%%%%%%%%%%%%%%%%%%%%%%%%%%%%%%%%%%%%

\section*{Introduction}
\label{section:introduction}

Molecular simulations use Markov chain Monte Carlo (MCMC) techniques~\cite{jun-s-liu:mcmc} to sample configurations $x$ from an equilibrium distribution $\pi(x)$, either exactly (using Monte Carlo methods) or approximately (using molecular dynamics simulations).

Due to the nature of the equilibrium distribution $\pi(x)$ and the difficulty in producing a sufficiently good guess of an equilibrium configuration, these molecular simulations are often started from highly atypical initial conditions.
For example, simulations of biopolymers might be initiated from a fully extended conformation unrepresentative of behavior in solution; solvated systems may be prepared by periodically replicating a small solvent box that was equilibrated with a different forcefield under different conditions from the current simulation, thus yielding atypical densities; 
liquid mixtures or lipid bilayers may be constructed by using methods that fulfill spatial constraints but create locally aytpical geometries (e.g. PackMol~\cite{martinez:jctc:2009:packmol}) that may require long simulation times to relax to typical configurations.

As a result, common practice in molecular simulations is to discard some initial portion of the trajectory to ``equilibration'' (also called \emph{burn-in}\footnote{The term \emph{burn-in} comes from the field of electronics, in which a short ``burn-in'' period is used to ensure that a device is free of faulty components---which often fail quickly---and is operating normally~\cite{crc-mcmc-handbook}.} in MCMC literature~\cite{crc-mcmc-handbook}).
While this is strictly unnecessary~\cite{geyer:burn-in-unnecessary,crc-mcmc-handbook}, this often allows the practitioner to avoid what would otherwise be extremely long run times to eliminate the bias from the initial atypical starting conditions in computed averages.

As an illustrative example of this effect, consider the simulation shown in {\bf Figure~\ref{figure:burn-in-example}}, in which a simulation of liquid argon is started at an atypically density and allowed to equilibrate to its equilibrium density (see caption for detailed description of simulation methods).
The expectation of the running average of the density over many realizations of this procedure ({\bf Figure~\ref{figure:burn-in-example}b}) significantly deviates from the actual expectation, which would lead to biased estimates unless simulations were sufficiently long to eliminate this starting point dependent bias.
Note that this significant bias is present because the \emph{same} atypical starting condition is used for every realization of this simulation process.

For the purposes of this note, we presume that the goal is to compute some form of equilibrium expectation, <A> from a timeseries average:
\begin{eqnarray}
\hat{A} \approx \int_0^T dt \, A(x(t))
\end{eqnarray}

%%%%%%%%%%%%%%%%%%%%%%%%%%%%%%%%%%%%%%%%%%%%%%%%%%%%%%%%%%%%%%%%%%%%%%%%%%%%%%%%
% FIGURE: BURN-IN EXAMPLE
%%%%%%%%%%%%%%%%%%%%%%%%%%%%%%%%%%%%%%%%%%%%%%%%%%%%%%%%%%%%%%%%%%%%%%%%%%%%%%%%

\begin{figure*} 
%\resizebox{0.8\textwidth}{!}{\includegraphics{burn-in-example.pdf}}
\caption{\label{figure:burn-in-example} {\bf Illustration of the motivation for discarding data to equilibration in computing expectations from molecular simulations.} 
This is text.}
\end{figure*}

%%%%%%%%%%%%%%%%%%%%%%%%%%%%%%%%%%%%%%%%%%%%%%%%%%%%%%%%%%%%%%%%%%%%%%%%%%%%%%%%
% METHODS
%%%%%%%%%%%%%%%%%%%%%%%%%%%%%%%%%%%%%%%%%%%%%%%%%%%%%%%%%%%%%%%%%%%%%%%%%%%%%%%%

\section*{methods}
\label{section:methods}

All molecular simulations were performed with OpenMM 6.2~\cite{eastman:jctc:2012:openmm} using the Python API.
All scripts used to run simulations, analyze data, and generate plots---along with the simulation data itself---are availabile on GitHub at \url{http://github.com/choderalab/automatic-equilibration-detection}.


%%%%%%%%%%%%%%%%%%%%%%%%%%%%%%%%%%%%%%%%%%%%%%%%%%%%%%%%%%%%%%%%%%%%%%%%%%%%%%%%
% BIBLIOGRAPHY
%%%%%%%%%%%%%%%%%%%%%%%%%%%%%%%%%%%%%%%%%%%%%%%%%%%%%%%%%%%%%%%%%%%%%%%%%%%%%%%%

\bibliographystyle{prsty} 
\bibliography{automatic-equilibration-detection}

\end{document}