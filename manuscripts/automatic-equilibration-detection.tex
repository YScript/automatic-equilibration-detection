% PRL look and style (easy on the eyes)
\documentclass[aps,pre,twocolumn,nofootinbib,superscriptaddress,linenumbers]{revtex4-1}
% Two-column style (for submission/review/editing)
%\documentclass[aps,prl,preprint,nofootinbib,superscriptaddress,linenumbers]{revtex4-1}

%\usepackage{palatino}

% Change to a sans serif font.
\usepackage{avant}
\renewcommand*\familydefault{\sfdefault} %% Only if the base font of the document is to be sans serif
\usepackage[T1]{fontenc}

\usepackage{amsmath}
\usepackage{amssymb}
\usepackage{graphicx}
%\usepackage[mathbf,mathcal]{euler}
%\usepackage{citesort}
\usepackage{dcolumn}
\usepackage{boxedminipage}
\usepackage{verbatim}
\usepackage[colorlinks=true,citecolor=blue,linkcolor=blue]{hyperref}


% The figures are in a figures/ subdirectory.
\graphicspath{{figures/}}


% italicized boldface for math (e.g. vectors)
\newcommand{\bfv}[1]{{\mbox{\boldmath{$#1$}}}}
% non-italicized boldface for math (e.g. matrices)
\newcommand{\bfm}[1]{{\bf #1}}          

%\newcommand{\bfm}[1]{{\mbox{\boldmath{$#1$}}}}
%\newcommand{\bfm}[1]{{\bf #1}}
\newcommand{\expect}[1]{\left \langle #1 \right \rangle}                % <.> for denoting expectations over realizations of an experiment or thermal averages

% vectors
\newcommand{\x}{\bfv{x}}
\newcommand{\y}{\bfv{y}}
\newcommand{\f}{\bfv{f}}

\newcommand{\bfc}{\bfm{c}}
\newcommand{\hatf}{\hat{f}}

\newcommand{\bTheta}{\bfm{\Theta}}
\newcommand{\btheta}{\bfm{\theta}}
\newcommand{\bhatf}{\bfm{\hat{f}}}
\newcommand{\Cov}[1] {\mathrm{cov}\left( #1 \right)}
\newcommand{\Ept}[1] {{\mathrm E}\left[ #1 \right]}
\newcommand{\Eptk}[2] {{\mathrm E}\left[ #2 \,|\, #1\right]}
\newcommand{\T}{\mathrm{T}}                                % T used in matrix transpose
\newcommand{\conc}[1] {\left[ \mathrm{#1} \right]}

\newcommand{\pyitc}{\url{http://www.simtk.org/home/bayesian-itc}} % URL of pyITC project homepage

%% DOCUMENT %%%%%%%%%%%%%%%%%%%%%%%%%%%%%%%%%%%%%%%%%%%%%%%%%%%%%%%%%%%%%%%%%%%%
\begin{document}

%% TITLE %%%%%%%%%%%%%%%%%%%%%%%%%%%%%%%%%%%%%%%%%%%%%%%%%%%%%%%%%%%%%%%%%%%%
\title{A simple method for automated equilibration detection in molecular simulations}

\author{John D. Chodera}
 \thanks{Corresponding author}
 \email{john.chodera@choderalab.org}
 \affiliation{Computational Biology Program, Sloan Kettering Institute, Memorial Sloan Kettering Cancer Center, New York, NY 10065}

\date{\today}

%%%%%%%%%%%%%%%%%%%%%%%%%%%%%%%%%%%%%%%%%%%%%%%%%%%%%%%%%%%%%%%%%%%%%%%%%%%%%%%%%%%%%%%%%%%%%%%%%%%%%%
% ABSTRACT/pacs
%%%%%%%%%%%%%%%%%%%%%%%%%%%%%%%%%%%%%%%%%%%%%%%%%%%%%%%%%%%%%%%%%%%%%%%%%%%%%%%%%%%%%%%%%%%%%%%%%%%%%%
\begin{abstract}
Molecular simulations (molecular dynamics, Monte Carlo) are often initiated from configurations that are highly dissimilar to equilibrium samples, a practice which causes the appearance of a distinct initial transient in various mechanical observables computed over the timecourse of the simulation.
Traditional practice in simulation data analysis recommends this initial transient portion be discarded as \emph{equilibration}, but no simple, general automated procedure exists.
Here, we consider a simple, automated, easy-to-implement procedure in which the final region of the simulation that maximizes the number of effectively uncorrelated samples is used.
We present a reference implementation of this procedure and illustrate its application to synthetic and real data.

\emph{Keywords: molecular dynamics (MD); Monte Carlo (MC); Markov chain Monte Carlo (MCMC); equilibration; timeseries analysis; statistical inefficiency; integrated autocorrelation time}
\end{abstract}

\maketitle

%%%%%%%%%%%%%%%%%%%%%%%%%%%%%%%%%%%%%%%%%%%%%%%%%%%%%%%%%%%%%%%%%%%%%%%%%%%%%%%%%%%%%%%%%%%%%%%%%%%%%%
% INTRODUCTION
%%%%%%%%%%%%%%%%%%%%%%%%%%%%%%%%%%%%%%%%%%%%%%%%%%%%%%%%%%%%%%%%%%%%%%%%%%%%%%%%%%%%%%%%%%%%%%%%%%%%%%
\section{Introduction}
\label{section:introduction}

% Molecular simulations generate correlated timeseries data; we can think of this as a realization of an MCMC chain
% Often, we want to compute an average (expectation) of some property A(x)
% Standard practice is to discard the initial transient to "equilibration" (a process known in the MCMC literature as "burn-in")
% In principle, if we run the chain long enough, it will converge to the true mean; no discarding is required (Gelman)
% In practice, however, starting from highly atypical initial conditions may mean that we need to run for a very long time
% Often, we're limited in how much data we can collect, so we are faced with the problem of how to make the best of the data we were able to collect

% Other schemes for determining the equilibrated region exist
% Some are based on statistical hypothesis tests, and assume the data follow a Gaussian form (Wei Yang, Karplus)

% We propose a simple approach: Select the origin of the equilibrated (production) region $t_0$ that maximizes the number of effectively uncorrelated samples

% Aside on computing the effective number of correlated samples and statistical error

%%%%%%%%%%%%%%%%%%%%%%%%%%%%%%%%%%%%%%%%%%%%%%%%%%%%%%%%%%%%%%%%%%%%%%%%%%%%%%%%%%%%%%%%%%%%%%%%%%%%%%
% BIBLIOGRAPHY
%%%%%%%%%%%%%%%%%%%%%%%%%%%%%%%%%%%%%%%%%%%%%%%%%%%%%%%%%%%%%%%%%%%%%%%%%%%%%%%%%%%%%%%%%%%%%%%%%%%%%%

%\bibliographystyle{prsty} 
\bibliography{bayesian-itc.bib}

\end{document}